\documentclass[a4paper]{exercisesheet}

%\usepackage{mathpazo}
%\usepackage{mathptmx}
%\usepackage{newtxmath}
\usepackage[T1]{fontenc}
\usepackage[utf8]{luainputenc}
\usepackage[charter]{mathdesign}
\usepackage{graphicx}


\let\sfdefault=\rmdefault
\def\ttdefault{txtt}

\usepackage[dvipsnames]{xcolor}
%\colorlet{maincolor}{blue!50!black}
\colorlet{maincolor}{black}

\usepackage{listings}
\usepackage{amsmath}
%\lstset{
%  frame=lines,
%  backgroundcolor=\color{maincolor!15},
%  rulecolor=\color{maincolor},
%  language=Python
%  keywordstyle=\bfseries\color{maincolor},
%  numbers=left,
%  numberstyle=\scriptsize\color{maincolor!70},
%}

\linespread{1.04}

\usepackage[english]{babel}

\usepackage{blindtext}

\sheetconf{
    lecture   = {Netzwerke und komplexe Systeme},
  lecturer  = {F.~Klimm~und~B.F.~Maier},
  semester  = {Sch\"ulerakademie 5.2 (Ro\ss leben 2016)},
  author    = {},
  % teacher,
  solutions=false,
}

\setsheetfont{lecture on titlepage}{\sffamily\Huge}
\setsheetfont{sheet title}{\sffamily\Large}
\setsheetfont{type on titlepage}{\sffamily\scriptsize\color{maincolor}}
\setsheetfont{sheet topic}{\sffamily\Huge\color{maincolor}}
\setsheetfont{sheet lecture}{\it\sffamily\Large\color{maincolor}}
\setsheetfont{exercise topic}{\sffamily\Large\color{maincolor}}
\setsheetfont{exercise label}{\sffamily\Large\color{maincolor}}
\setsheetfont{subexercise topic}{\sffamily\large\color{maincolor}}
\setsheetfont{subexercise label}{\sffamily\large\color{maincolor}}

\setsheettemplate{sheet title (student)}{Übungsblatt~\thesheet}
\setsheettemplate{exercise name}{Aufgabe}
\setsheettemplate{subexercise name}{Teilaufgabe}

\lstset{%
  %linewidth=\textwidth,
  %linewidth=16cm,
  language=Python,                  % the language of the code
  basicstyle=\ttfamily\small,
  backgroundcolor=\color{maincolor!5},
  %basicstyle=\footnotesize,      % the size of the fonts that are used for the code
  numbers=left,                   % where to put the line-numbers
  stepnumber=1,                   % the step between two line-numbers. If it's 1, each line 
                                  % will be numbered
  numberstyle=\scriptsize\color{maincolor!70},
  numbersep=5pt,                  % how far the line-numbers are from the code
  frame=single,                   % adds a frame around the code
  rulecolor=\color{black},        % if not set, the frame-color may be changed on line-breaks
  tabsize=4,                      % sets default tabsize to 2 spaces
  captionpos=b,                   % sets the caption-position to bottom
  breaklines=true,                % sets automatic line breaking
  breakatwhitespace=false,        % sets if automatic breaks should only happen at whitespace
  %keywordstyle=\color{blue},      % keyword style
  %commentstyle=\color{dkgreen},   % comment style
  %stringstyle=\color{colorNavy},  % string literal style
  morekeywords={*,with, where, from, union, all, as},
  extendedchars=true,
  literate={ä}{{\"{a}}}1 {ö}{{\"o}}1 {ü}{{\"u}}1,
}


\begin{document}

  \sheet[%
  number=1,
      topic={Einf\"uhrung in Netzwerktheorie},
      %deadline=Deadline: \today,
    ]

\vspace{-1cm}
\noindent\rule{12cm}{0.4pt}

  \exercise[%
  topic = Graphen mit Python erstellen.
    ]

Wir wollen die Programmiersprache Python nutzen um Graphen zu erstellen
und zu untersuchen. Die folgende Funktion {\tt pfadgraph} in der Datei {\tt pfadgraph.py} erzeugt zum Beispiel die Adjazenzmatrix eines Pfadgraphen.
  
\lstinputlisting{./code/pfadgraph.py}



 \subexercise[%
  topic=Programm verstehen und verändern,
    ]

Gehe Zeile für Zeile über den Programm-Code um ihn zu verstehen. Beantworte insbesondere die folgenden Fragen:

\begin{enumerate}
\item Was sind In- und Output der Funktion {\tt pfadgraph( n )}?
\item Wie sieht die Adjazenzmatrix in jedem Schritt der FOR-Schleife aus? Überprüfe deine Vermutung indem du nach Zeile $8$ die folgende Zeile einfügst {\tt print(A)}
\item Warum geht die FOR-Schleife von $0$ bis $n-1$?
\item Was passiert in Zeile $10$? Wie würde die Adjazenzmatrix aussehen wenn diese Zeile nicht eingefügt wäre? Überprüfe deine Vermutung indem du die Zeile auskommentierst.
\end{enumerate}

Das gute an Funktionen ist, dass wir sie wieder und wieder verwenden können. Erweitere das obige Programm, so dass dir nach der Adjazenzmatrix eines Pfades bestehend aus 10 Knoten auch den für 15 Knoten ausgibt.\\

 \subexercise[%
  topic=Netzwerke erstellen,
    ]

Nutze das obige Programm als Ausgangspunkt um Funktionen zu schreiben die die folgenden Graphen erzeugen:

\begin{enumerate}
\item Nullgraph
\item Kreisgraph
\item Vollständiger Graph
\item Vollständiger Bipartiter Graph mit Gruppengrössen $m$ und $n$
\end{enumerate}




  %\subexercise[%
  %topic=Netzwerke erstellen,
    %]


\exercise[%
  topic=Netzwerke mit Python darstellen
    ]

Wir erstellen Graphen meist als Adjazenzmatrix. Diese sind nützlich um die sie weiter mathematish zu analysieren. Manchmal ist es allerdings von Vorteil ein Netzwerk darzustellen. Hierzu können wir zum Beispiel die Bibliothek {\tt networkx} benutzen:  

\lstinputlisting{./code/graph_darstellen.py}

Wir müssen nur die nötigen Bibliotheken laden und die Adjazenzmatrix in ein anderes Objekt umwandeln. Dieses kann dann von der Funktion {\tt draw()} genutzt werden um eine Darstellung zu erzeugen.\\
Nutze {\tt networkx} um verschiedene Graphen der Grösse $n=8$ darzustellen.

  \exercise[%
  topic=Netzwerke mit Python untersuchen
    ]
Nun werden wir die in Aufgabe $1$ erstellten Funktionen nutzen um diese Netzwerke zu untersuchen. Nutze die Funktion {\tt loadtxt} um eine Adjazenzmatrix aus einer Textdatei zu laden.

\begin{lstlisting}
H = np.loadtxt("hennen.txt")
\end{lstlisting}

Bei dem Netzwerk handelt es sich um ein Netzwerk das angibt welche von $32$ Hühner miteinander kämpfen.\\
Zähle die Kanten, stelle das Netzwerk dar und vergleiche mit den
Netzwerkmodellen die wir eingeführt haben. Welches der Modelle
beschreibt diese Daten am besten?\\
Wiederhole dies mit einem zweiten Netzwerk {\tt florence.txt}. Ist eines
der Modelle in der Lage die Daten gut zu beschreiben?\\
Dieses Netzwerk beschreibt die ehelichen Beziehungen von Adelsfamilien
im Florenz des 13. Jahrhunderts. In der Datei 
   {\tt florence\_families.txt} 
   findet ihr die Namen der Adelsgeschlechter (Zeilennummer
   korrespondiert mit Knotennummer im Netzwerk). Welche Familie hat die
   meisten Verbindungen?
{\tt

 \subexercise[%
  topic=Kanten zählen,
    ]

Wir wollen untersuchen wie die Anzahl $m$ von Kanten im Netzwerk sich in Abhängigkeit von der Anzahl der Knoten $n$ verhält. Führe dafür die folgenden Schritte aus:

\begin{enumerate}
\item Schreibe eine Funktion die als Input eine Adjazenzmatrix und als Output die Anzahl von Kanten hat. (\emph{Hinweis: Die numpy-Funktion sum() könnte nützlich sein.})
\item Nutze eine FOR-Schleife um Netzwerke variierender Grösse $n$ zu erstellen und deren Kanten zu zählen.
\item Nutze maplotlib um diese Messungen darzustellen. 
\item Vergleiche diese Werte mit den am Vormittag ermittelten analytischen Formeln.
\end{enumerate}


 \subexercise[%
  topic=Vergleich mit empirischen Netzwerken,
    ]

Bis jetzt haben wir uns nur mit Netzwerkmodellen beschäftigt. Nun wollen wir diese real existierenden (empirischen) Netzwerke vergleichen.


\exercise[%
  topic=Zusatzaufgaben
    ]

\subexercise[%
  topic=Multipartite Netzwerke,
    ]
		
    Ähnlich wie die bipartiten Netzwerke mit zwei Gruppen von Knoten
    gibt es auch multipartite (oder $p$-partite) Netzwerke mit meheren
    Gruppen. Schreibe eine Funktion die solche $p$-partiten Netzwerke
    erzeugt wobei jede der $p$ Gruppen die gleiche Grö\ss e $g$ besitzt.\\
    Finde analytische Formeln für die Anzahl $n$ der Knoten im gesamten
    Netzwerk und der Anzahl $m$ der Kanten und überprüfe sie numerisch.
		
		\subexercise[%
  topic=Knoten mit gleichem Grad,
    ]
		Beweise, dass jeder Graph $G$ der mehr als einem Knoten mindestens zwei Knoten mit gleichem Grad besitzt.
 
\subexercise[%
  topic=$3$-reguläre Graphen,
    ]
		Ein Graph ist $r$-regulär wenn jeder Knoten den Grad $r$ besitzt. Beweise, dass es genau dann einen $3$-regulären Graphen mit $n$ Knoten gibt, wenn $n>3$ und $n$ gerade ist.
		
		
		\subexercise[%
  topic=Dijkstra-Algorithmus,
    ]
		
		Implementieren den Dijkstra-Algorithmus. Überprüfe ob seine Zeitkomplexität $\mathcal{O}(N^2)$
		
		%\exercise[%
  %topic=Netzwerke mit Python untersuchen
    %]

\end{document}
