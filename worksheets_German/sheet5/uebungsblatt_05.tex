\documentclass[a4paper]{exercisesheet}

%\usepackage{mathpazo}
%\usepackage{mathptmx}
%\usepackage{newtxmath}
\usepackage[T1]{fontenc}
\usepackage[utf8]{luainputenc}
\usepackage[charter]{mathdesign}
\usepackage{graphicx}


\let\sfdefault=\rmdefault
\def\ttdefault{txtt}

\usepackage[dvipsnames]{xcolor}
%\colorlet{maincolor}{blue!50!black}
\colorlet{maincolor}{black}

\usepackage{listings}
\usepackage{amsmath}
%\lstset{
%  frame=lines,
%  backgroundcolor=\color{maincolor!15},
%  rulecolor=\color{maincolor},
%  language=Python
%  keywordstyle=\bfseries\color{maincolor},
%  numbers=left,
%  numberstyle=\scriptsize\color{maincolor!70},
%}

\linespread{1.04}

\usepackage[english]{babel}

\usepackage{blindtext}

\sheetconf{
    lecture   = {Netzwerke und komplexe Systeme},
  lecturer  = {F.~Klimm~und~B.F.~Maier},
  semester  = {Sch\"ulerakademie 5.2 (Ro\ss leben 2016)},
  author    = {},
  % teacher,
  solutions=false,
}

\setsheetfont{lecture on titlepage}{\sffamily\Huge}
\setsheetfont{sheet title}{\sffamily\Large}
\setsheetfont{type on titlepage}{\sffamily\scriptsize\color{maincolor}}
\setsheetfont{sheet topic}{\sffamily\Huge\color{maincolor}}
\setsheetfont{sheet lecture}{\it\sffamily\Large\color{maincolor}}
\setsheetfont{exercise topic}{\sffamily\Large\color{maincolor}}
\setsheetfont{exercise label}{\sffamily\Large\color{maincolor}}
\setsheetfont{subexercise topic}{\sffamily\large\color{maincolor}}
\setsheetfont{subexercise label}{\sffamily\large\color{maincolor}}

\setsheettemplate{sheet title (student)}{Übungsblatt~\thesheet}
\setsheettemplate{exercise name}{Aufgabe}
\setsheettemplate{subexercise name}{Teilaufgabe}

\lstset{%
  %linewidth=\textwidth,
  %linewidth=16cm,
  language=Python,                  % the language of the code
  basicstyle=\ttfamily\small,
  backgroundcolor=\color{maincolor!5},
  %basicstyle=\footnotesize,      % the size of the fonts that are used for the code
  numbers=left,                   % where to put the line-numbers
  stepnumber=1,                   % the step between two line-numbers. If it's 1, each line 
                                  % will be numbered
  numberstyle=\scriptsize\color{maincolor!70},
  numbersep=5pt,                  % how far the line-numbers are from the code
  frame=single,                   % adds a frame around the code
  rulecolor=\color{black},        % if not set, the frame-color may be changed on line-breaks
  tabsize=4,                      % sets default tabsize to 2 spaces
  captionpos=b,                   % sets the caption-position to bottom
  breaklines=true,                % sets automatic line breaking
  breakatwhitespace=false,        % sets if automatic breaks should only happen at whitespace
  %keywordstyle=\color{blue},      % keyword style
  %commentstyle=\color{dkgreen},   % comment style
  %stringstyle=\color{colorNavy},  % string literal style
  morekeywords={*,with, where, from, union, all, as},
  extendedchars=true,
  literate={ä}{{\"{a}}}1 {ö}{{\"o}}1 {ü}{{\"u}}1,
}


\usepackage{pstricks,pst-node,pst-tree}


\begin{document}

  \sheet[%
  number=5,
  topic={Dynamische Systeme},
      %deadline=Deadline: \today,
    ]

\vspace{-1cm}
\noindent\rule{12cm}{0.4pt}

  \exercise[%
  topic = Populationsdynamik in diskreter Zeit
  ]

  In dem Vortrag am Vormittag haben wir die Populationsdynamik von Verhulst
 diskutiert, deren Abbildungsgleichung lautet

  \begin{equation}
      x_{n+1} = \lambda x_n(1-x_n)\,.
  \end{equation}

  \begin{enumerate}
      \item Wenn $x\in [0,1]$ gelten soll,  wie ist dann die
          Beschr\"ankung des Wachstumsparameters $\lambda$?
      \item Was sind die Fixpunkte $x^*$ der Dynamik?
      \item Wie ist die Stabilit\"at des kleineren der zwei
          Fixpunkte f\"ur $\lambda<1$? Wie ist sie f\"ur $\lambda>1$?
          Um das zu \"uberpr\"ufen, setze Werte $x^*+\epsilon$ ein (mit
          $\epsilon\ll1$). Der Wert $\epsilon$ ist eine so genannte
          \textit{St\"orung}. Um die Stabilit\"at des Fixpunktes 
          zu ermitteln, untersuche, wie sich die Abbildung
          verh\"alt f\"ur eine kleine St\"orung (z.B. $\epsilon=0.01$).
  \end{enumerate}

    
  \exercise[%
  topic = Dynamische Systeme in kontinuierlicher Zeit
  ]

  Dynamische Systeme in kontinuierlicher Zeit sind bestimmt durch
  \begin{equation}
      \dot x = f(x)\,,
  \end{equation}
  wobei die Funktion $f$ eine Abbildung $f:X\rightarrow X$ ist. In
  diesen Systemen sind Fixpunkte $x^*$ gegeben durch
  \begin{equation}
      f(x^*) = \dot x|_{x=x^*} = 0\,.
  \end{equation}
  \subexercise
  Betrachte das dynamische System
  \begin{equation}
      \dot x = x^2 -1\,.
  \end{equation}
  \begin{enumerate}
      \item Bestimme die Fixpunkte.
      \item Bestimme die Stabilit\"at der Fixpunkte mit der
          Vektorfeldmethode.
  \end{enumerate}

  \subexercise
  Bestimme die Fixpunkte und die Stabilit\"at der dynamischen Systeme
  \begin{enumerate}
      \item $\dot x = - x^3$
      \item $\dot x = x^3$
      \item $\dot x =x^2$
      \item $\dot x = x$
      \item $\dot x = 0$
      \item $\dot x = x-x^3$
  \end{enumerate}
  
  \subexercise[topic=Bifurkation]
  Betrachte das dynamische System
  \begin{equation}
      \dot x = x^2 - a\,.
  \end{equation}
  mit $a\in I\!\!R$.
  Bestimme die Fixpunkte und ihre Stabilit\"at. Wieviele
  Fixpunkte gibt es f\"ur verschiedene Werte von $a$?

  \subexercise[topic=Newton-Reibung]
  Die Reibungskraft der Luft auf fallende Gegenst\"ande (auf der Erde,
  mit Fallbeschleunigung $g$) kann beschrieben
  werden durch  $F(v) = \beta v^2$, wobei $\beta$ eine Reibungskonstante
  ist und $v$ die Geschwindigkeit der zu betrachtenden Masse $m$. Die
  Kraft wirkt in positive $z$-Richtung, also dem Fallen entgegen. Die
  Zeitentwicklung der Geschwindigkeit ist dann gegeben als
  \begin{equation}
      m\dot v = -mg+\beta v^2\,.
  \end{equation}
  Was ist der Fixpunkt $v^*$? Ist er stabil? Wie ist der Fixpunkt physikalisch zu
  interpretieren? Warum ist das Ergebnis der Stabilit\"atsanalyse
  intuitiv?

  Die allgemeine Form dieser Zeitentwicklung von $v$ ist
  \begin{equation}
      m\dot v = -mg-\mathrm{sgn}(v)\beta v^2
  \end{equation}
  mit der Vorzeichenfunktion
  \begin{equation}
      \mathrm{sgn}(x)=\begin{cases}-1 & x<0\\
          0 & x=0\\
          +1 & x>0\end{cases}\,.
  \end{equation}
  Das hei\ss{}t, die Reibung wirkt der Bewegung immer entgegen.
  Bestimme nun das Gleichgewicht $v^*$ f\"ur $g=0$. Ist der Fixpunkt
  stabil? \"Uberlege dir eine physikalische Situation des Systems mit
  $g=0$. Ergibt die Stabilit\"atsanalyse ein sinnvolles Ergebnis?


  \exercise[topic=Lineare Stabilit\"at]
  \subexercise
  Betrachte das System
  \begin{equation}
      \dot x = \lambda x
  \end{equation}
  mit $\lambda\in I\!\!R$. Welches nat\"urliche System ist durch diese Gleichung beschrieben?

  L\"ose die Gleichung 
  mit der Methode der Trennung der Variablen (verwende dazu deine Aufzeichnungen
  von der Krankheitsausbreitung). 
  Hat das System einen Fixpunkt? Falls ja, f\"ur welche Werte von
  $\lambda$? 

  \subexercise
  Betrachte eine beliebige, differenzierbare Funktion $f(x)$. Zeige,
  dass die Funktion an einer beliebigen Stelle $x_0$ linear approximiert werden
  kann als
  \begin{equation}
      f(x) \approx f(x_0) + m(x-x_0)\,.
  \end{equation}
  Zeige, dass die Steigung $m$ bestimmt werden kann als $m=f'(x_0)$ mit
  der ersten Ableitung $f'(x)$.

  \subexercise
  Zeige ohne grafische Methoden, dass f\"ur das dynamische System 
  \begin{equation}
      \dot x = x-x^3
  \end{equation}
  die Stellen $x^*_1=-1$ und $x^*_3=1$ linear stabil sind.

  \exercise[topic=Eigenwerte und Eigenvektoren]
  \subexercise
  Betrachte die Matrix
  \begin{equation}
      \hat A = \begin{pmatrix} 1 & 1 \\ 0 & 2\end{pmatrix}\,.
  \end{equation}
  Was sind die Eigenwerte? Wie lauten die Eigenvektoren $\vec w_1$ und
  $\vec w_2$?
  
%  \subexercise
%  Seien die Koordinaten $(x,y)$ die Beschreibung des zweidimensionalen
%  Raumes, der durch die Vektoren 
%  \begin{equation}
%      \vec e_1 = \begin{pmatrix} 1\\0\end{pmatrix}\qquad
%      \mathrm{und}\qquad 
%      \vec e_2 = \begin{pmatrix} 0\\1\end{pmatrix}
%  \end{equation}
%  aufgespannt wird. Gegeben sei nun die Matrix 
%  \begin{equation}
%      \hat B = \begin{pmatrix} 1 & 2 \\ 2 & 1\end{pmatrix}
%  \end{equation}
%  aus der Vorlesung. Zeige, dass in den Koordinaten $(\tilde x,\tilde
%  y)$, die nat\"urlich f\"ur die Matrix $\hat B$ sind (d.h. in Form der
%  Eigenvektoren der Matrix ausgedr\"uckt werden k\"onnen), gilt
%  \begin{document}
%  \begin{pmatrix} \tilde x\\ \tilde y \end{pmatrix}
%  = \begin{pmatrix} 1\\  0 \end{pmatrix} + \begin{pmatrix}
%      \tilde x\\ \tilde y \end{pmatrix}
\end{document}
