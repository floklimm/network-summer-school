\input{./../tex_files/preamble_engl}

\usepackage{pstricks,pst-node,pst-tree}



\begin{document}

  \sheet[%
  number=4,
      topic={Contagion Dynamics},
      %deadline=Deadline: \today,
    ]

\vspace{-1cm}
\noindent\rule{12cm}{0.4pt}

  \exercise[%
  topic = A Network of the Monastery School Ro\ss leben 
    ]

\label{rossleben}

The aim of this task is the creation of a network that represents the
building of the Monastery School Ro\ss leben. The network's rooms are
nodes and a pair of nodes is connected with an edge if there is a
\emph{direct} connection between them. In the end the network should be
available in a file as an adjacency matrix or edge list such that it can be loaded into
python and further analysed.

 \subexercise[%
  topic=Dividing the Problem up,
    ]

Discuss in a large group how the problem could be divided such that many
of you can work in parallel on it. In the end there should be one file
with the whole network. Create also a second file that gives the name of
the rooms, such as:


\begin{tabular}[h]{c|c|c}
Node number & floor & room description \\ \hline
1 & 0 & Administration Office\\
2 & 1 & Hall\\
\multicolumn{2}{c}{\vdots}
\end{tabular}



 \subexercise[%
  topic= Degree distribution,
    ]

Extract the degree distribution $P(k)$ of the network. Illustrate it as a histogram.


\exercise[%
  topic =  Pandemic in the Monastery School
    ]

The Course 5.3 `Germs and Diseases' is experimenting with infectious germs. Simulate the spread of a disease that starts in their laboratory. Use the SI model dynamics on the network created in Exercise~\ref{rossleben}.


\subexercise[%
  topic= Complete Infection,
    ]


Simulate the contagion process for varying infection rates $\beta \in [0,1]$. After how many time steps $t_{\mathrm{end}}$ is the whole network infected for the different rates $\beta$? Illustrate the results for $t_{\mathrm{end}}(\beta)$ in a figure and interpret your findings.

Optional:  Average $t_{\mathrm{end}}(\beta)$ for $s=20$ simulations. How does this change the curve?

\subexercise[%
  topic= Vulnerable Courses,
    ]

The contagion again starts from laboratory of course $5.3$. We want to
determine which of the other five courses is most vulnerable to the
contagion. Simulate the SI-Process for $\beta=0.01$ and investigate at
which point in time the rooms of the other courses and the
Administration office are infected. 

Which rooms are most vulnerable? How can you explain these results?


\exercise[%
  topic =  \emph{Betweenness} Centrality
      ]


In addition to the degree $k_i$ of a node there exist also other
measures of centrality of a node, for example, the \emph{betweenness}
centrality $g_i$. It measures in how many shortest paths between all
pairs of nodes a certain node $i$ exists. To be specific it is defined as 
\begin{align}
g_i = \sum_{s\neq i\neq t} \frac{\sigma_{st}(i)}{\sigma_{st}}\,,
\end{align}
where $\sigma_{st}$ is the number of shortest paths between nodes $s$
and $t$ and  $\sigma_{st}(i)$ the number of these paths that go through
node $i$. For finding all shortest paths between nodes $s$ and $t$ in a
network $G$ you can use the function
\texttt{networkx.all\_shortest\_paths(G,s,t)}.

\subexercise[%
  topic= Betweenness Centrality of Graph Models,
    ]

\begin{itemize}
\item Sketch a path graph $P_n$ and find $g_i\,\forall\ i \in V$ in dependence of the number $n$ of nodes.
\item Calculate the betweenness $g_i$ for the cycle graph $C_n$.
\item Which two graphs with $n$ nodes have $g_i=0\,\forall\ i \in V$? Show that both are not isomorph to each other if $n>1$.
\end{itemize}


\subexercise[%
  topic= Betweenness Centrality of a Complete Graph after Deletion of an Edge,
    ]

Be $K_n$ the complete graph with $n$ nodes and $e=\{u,v\}$ one of its edges. Show that the betweenness centrality of the graph  $K_n - e$ is then given by
		
\begin{align}
g_i = \begin{cases} \frac{1}{n-2} &\mbox{if } i \in \{u,v\} \\ 
0 & \mbox{else } \end{cases} 
\end{align} 
betr\"agt.

\subexercise[%
  topic= Betweenness Centrality of two connected $n$-Cliques,
    ]


The graph $G$ consists of two $n$-cliques that are connected via a single node that connects to all other nodes. Find formulas for the betweenness $g_i$ of this connection node $i$.

Optional: Generalise this for $c$ of such $n$-cliques that are connected through that single central node and find the function  $g_i(c,n)$.

%\subexercise[%
  %topic= Betweenness Zentralit\"at Programmieren,
    %]		
%
%Schreibe ein Programm welches 
		
\exercise[%
  topic =  Contagion Dynamic -- Vaccination
    ]

We want to investigate how the contagion process is slowed down if we vaccinate certain rooms. We achieve this by removing these rooms from the network.

\subexercise[%
  topic= Random Vaccination,
    ]
\label{impfung}


At first we want to vaccinate the rooms randomly. Vaccinate a single
room and investigate whether the time $t_{\mathrm{end}}$ to the total
infection is changed. Repeat this experiment a couple of times. Does the
behaviour change under this repetitions? Interpret the result.


\subexercise[%
  topic= Random Next Neighbour Vaccination,
    ]

Now we will repeat the random vaccination but use our insights from the
\emph{friendship paradox}. We do not vaccinate a random room but a
random neighbour of a random room. Repeat the experiments and
investigate whether the behaviour is changed notably.

\subexercise[%
  topic= Systematic Investigation of the Vaccination Success,
    ]

After these initial observations we will analyse the impact of vaccinations systematically. For this we vaccinate not only a single room but iteratively rooms, at first one, then a second, and so on, until we removed half of all rooms. For each of these iterative steps we calculate:
\begin{itemize}
\item The number $N_{1}$ of nodes that are infected after a single step.
\item The number $N_{10}$ of nodes that are infected after ten steps.
\item The number $N_{100}$ of nodes that are infected after a hundred steps.
\item The number $N_{1000}$ of nodes that are infected after a thousand steps.
\end{itemize}


Vary the infection rate $\beta \in \{0.01,0.1,0.5,1\}$.

Note that you can stop the dynamic when all nodes are infected. For this you can use the python commands {\tt if} and {\tt break}.

Repeat the experiments with the next neighbour vaccination strategy. 

Interpret your results and discuss which strategy is more efficient. Why is this the case? Under which conditions will the whole network be infected?


\subexercise[%
  topic= Optional: Vaccination by \emph{Betweenness} Centrality,
    ]
    
    Repeat the experiments for a vaccination strategy such that you vaccinate nodes in decreasing order of their betweenness centrality $g_i$. How does the contagion dynamic change under this targeted vaccination strategy?
   

%		\exercise[%
%  topic =  Zusatz: Kaskaden-Modell
%    ]

%Neben dem SI-Modell f\"ur Krankheitsausbreitungen gibt es auch das sogenannte Kaskadenmodel f\"ur die Ausbreitung sozialer Ph\"anomene, zum Beispiel 

\end{document}
