% !TEX root = ./sheet_06.tex
\exercise[
    topic = Fractals and Self-similarity
]


In the morning we discussed the \emph{Koch curve} from two perspectives, as a Lindenmayer system (L-system) and as a fractal. Now we want to draw this fractal by iteratively using the L-system steps:
\begin{subequations}
\label{eq:Koch-L}
\begin{align}
    \mathrm{Variablen} &= \left\{F\right\} \\
    \mathrm{Konstanten} &= \left\{+,-\right\} \\
    \mathrm{Produktionsregeln} &= \left\{F\rightarrow F-F++F-F\right\} \\
    \mathrm{Axiom} &= F\,.
\end{align}
\end{subequations}

The $F$ indicates that the \emph{turtle} (which is a virtual drawing pen) moves {\bf F}orward, $-$ that it rotates by $60^\circ$ leftwards, and $+$ that it rotates $60^\circ$  rightwards.

\subexercise[
    topic = Iterative Application of the L-system
]
    \label{ex:L-Koch}
    
    
    We want to write a python function that applies the L-system rules iteratively $n$ times such that we receive a set of instructions that tells the turtle to draw a Koch curve. The \emph{axiom} is the initial condition.
    
\subexercise[
    topic = Teenage Mutant L-System Turtle
    ]

The following programme is an introdution into the drawing with \textit{turtle}. \footnote{A detailed documentation of the \textit{turtle}-commands is available under
        \url{https://docs.python.org/2/library/turtle.html}
    }

Change the commands in the first par of the programme and discuss how it changes the behaviour. What does the function in the end of the code achieve?
    \lstinputlisting{./code/turtleexample.py}

\subexercise[
    topic = Drawing a Koch curve
    ]


Write a function that gives the result of a $n$-fold iteration of the L-Systems (\ref{eq:Koch-L}) to a \emph{turtle} and then draws it. Note that the forward movement $F$ has to be scaled by the number $n$ of iterations, specifically with $1/3^n$.


How do we have to change the axiom such that not the Koch curve but the Koch snowflake is produced?

\subexercise[
    topic = General L-Systems
]
\label{ex:L-allg}

Generalise the function from Exercise~\ref{ex:L-Koch}, such that arbitrary L-systems can be produced and iterated $n$-fold.


\exercise[
    topic = Creating Fractals
]


We discuss the  L-system
\begin{subequations}
    \label{eq:Sierpinski-L}
\begin{align}
    \mathrm{Variablen} &= \left\{A,B\right\} \\
    \mathrm{Konstanten} &= \left\{+,-\right\} \\
    \mathrm{Produktionsregeln} &= \left\{A\rightarrow +B-A-B+,\
        B\rightarrow -A+B+A-\right\} \\
    \mathrm{Axiom} &= A\,.
\end{align}
\end{subequations}

Here, $A$ and $B$ indicate that the turtle is moving forward. The commands $-$ and $+$ represent left and right rotations by  $60^\circ$.

Which fractal is created?

\subexercise[  
topic = Optional: L-Systems with Memory
]

L-systems with memory mean that the turtle saves the current position in two-dimensional space when the command $[$ occurs. It then follows the commands inside the bracket until the command $]$. Then it `jumps' to the saved position from $[$ and follows further commands. 

We discuss the L-system 
\begin{subequations}
    \label{eq:Farn-L}
\begin{align}
    \mathrm{Variablen} &= \left\{F\right\} \\
    \mathrm{Konstanten} &= \left\{+,-,[,]\right\} \\
    \mathrm{Produktionsregeln} &= \left\{ F \rightarrow
        F[+F]F[-F]F\right\}\\
\mathrm{Axiom} &= F\,.
\end{align}
\end{subequations}

$F$ is again a forward movement and $+,-$ indicate rotations by $180^\circ/7$. What does this $L$-system resembles? 
